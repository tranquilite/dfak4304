\kapittelhode{Ekstraluminesensens 2. pillarteorem}
\subsection{Coda}

\begin{displaymath}
\left.\begin{aligned}
\text{for a given matrix V, of}~ m \times n, \\
\exists x, x \times V : \forall ~\text{element}(v \in V), \\
det(V) \bmod{2} = 0
\end{aligned}\right\} \qquad\text{iff} \quad v \in \mathbb{R}
\end{displaymath}
\vspace{2em}

eller, litt annerledes formulert;\\
fordi $ \exists x \models \exists \{ x : x \in S~(~\land x \in \mathbb{R} ) \} $, er vi tvunget til å undersøke hvorvidt det finnes flere mengder som tilfredsstiller de samme kravene.

\subsection{Bevisføring}
Det 2. pillarteoremets bevis er ofte brukt som utgangspunktet for videre studier, da det teoremet lar seg \emph{forholdsvis} lett avlede. Langformen av beviset involverer 23 mellombevis, men å undersøke alle 23 er ikke nødvendig for å oppnå forståelse for problemet og løsningen. Vi vil hovedsakelig befatte oss med 1., 3., 8., 14., og 22. Siden studenten allerede bør være kjent med grunnleggende rekursjon og induksjon, er det ikke nødvendig å dekke 8-14, men bevis 2. er, for de vitenbegjærende, inkludert i vedhenget på slutten av kompendiet.
