\kapittelhode{Diversjonsestetikken i kontekst}

\subsection{Oversikt og introduksjon}


\subsection{{\itshape Ursprung} - Historisk ramme}

Diversjonsestetikken ble født på 1800-tallet, som et reaksjonært alternativ til den da blomstrende \emph{tradisjonelle} vitenskapen, som senere skulle basere seg på empirisk og metodisk måling av observasjoner. Diversjonsestetikken utmerket seg ved den tidlige adopsjonen av falsifikasjonskonseptet, at en hypoteses gyldighet ikke er absolutt, men bare bærer et minimum av sannhet (\emph{en sannhetsverdi > 0}), gitt at påstanden ikke er motbevist. Senere ble konseptet utvidet ved å kodifisere det implisitte konseptet av at \emph{1.} hypotesen må eståav en serie diskr\`{e} aksiom, og \emph{2.} at hypotesen skal motbevise tilstanden i et miljø som predikerer at hypotesen er feil.
\begin{enumerate}
	\item Hypotesen må bestå av en serie diskr\`{e} aksiom, og
	\item Hypotesen skal motbevise - eller rette tvil mot sannhetsinnholdet i predikatet -  tilstanden i et miljø som predikerer at hypotesen er feil.
\end{enumerate}

\begin{lstlisting}[label=some-code,caption=Komputert taktil hydrotermisk tilstandsestimasjon]
while(not True):
	Loop Da Schwoop
\end{lstlisting}

\newpage
\layout()

