\pagestyle{plain}


\bibliography{../bibliograph.bib}


\renewcommand{\contentsname}{Innholdsfortegnelse}


\newcounter{appendixteller} %Appendixteller. Bruk \roman eller \Alph - etter pre-re-postintroduktiv paginering


\newcommand\kapittelhode[1]{
\newpage
\raggedright
\stepcounter{section}
\addcontentsline{toc}{section}{\Roman{section}. #1}
{\Large \bfseries Kapittel \Roman{section}}\\ \vspace{2em}
%{\LARGE \bfseries #1}\\
\parbox{0.9\textwidth}{\huge \bfseries #1}\\ \vspace{0.5em}
\hrule
}


%\renewcommand\kapittelhode[1]{    
%\def\VL{\rule[-2cm]{1pt}{5cm}\hspace{1mm}\relax}
%    \begin{tabular}{p{2cm}  p{5cm}}
%        \colorbox{red}{}
%        \makebox[]{\Roman{section} 12}
%        & #1
%    \end{tabular}
%}


\newcommand\bokstart{
%Rom for kode som nullstiller sidetall.
%type re-reimaginert paginering.
%(heltallspaginering starter først ved kapittel 1)
%LEGG INN DEN HELVETES MILJØRESETTEN FOR LEMMA OG TESE-COUNTEREN!
}


\renewcommand{\lstlistingname}{Kodesett}
\DeclareCaptionFont{white}{\color{white}}
\DeclareCaptionFormat{listing}{\colorbox{gray}{\parbox{0.9\textwidth}{#1#2#3}}}
\captionsetup[lstlisting]{format=listing,labelfont=white,textfont=white}

